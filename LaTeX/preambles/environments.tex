\ifalgorithms
    \newcounter{nalg}[chapter]
    \renewcommand{\thenalg}{\thechapter.\arabic{nalg}}
    \DeclareCaptionLabelFormat{algocaption}{\it Algorithm \thenalg}

    \lstnewenvironment{algorithm}[1][]
    {
        \refstepcounter{nalg}
        \captionsetup{labelformat=algocaption,labelsep=colon}
        \lstset{
            mathescape=true,
            frame=tB,
            numbers=left,
            numberstyle=\tiny,
            basicstyle=\scriptsize,
            keywordstyle=\color{black}\bfseries\em,
            keywords={,input, output, return, datatype, function, in, if, else, elif, for, foreach, while, not, begin, end, true, false, null, break, continue, let, and, or, }
            numbers=left,
            xleftmargin=.04\textwidth,
            #1
        }
    }
    {}
\else
\fi

% % new proof environment
% \expandafter\let\expandafter\oldproof\csname\string\proof\endcsname
% \let\oldendproof\endproof
% \renewenvironment{proof}[1][\proofname]{%
%     \vspace{-0.5\parskip}%
%     \oldproof[#1]
% }{%
%     \renewcommand{\qedsymbol}{\sc q.e.d.}
%     \qed
% }

% new proof environment
\expandafter\let\expandafter\oldproof\csname\string\proof\endcsname
\let\oldendproof\endproof
\renewenvironment{proof}[1][\bfseries\proofname]{%
    % \vspace{-0.5\parskip}%
    \oldproof[#1]
}{%
    ~\\\qed
}

% theorems
\declaretheoremstyle[
    headfont=\bfseries\itshape,
    bodyfont=\normalfont,
    notefont=,
    % headpunct={.\hfill\shifttext{2ex}{$\urcorner$}\break},
    headpunct={.\hfill\break},
    % headformat={\makebox[0pt][r]{\NAME\ \NUMBER}\NOTE},
    headformat={Thm \NUMBER \NOTE},
    qed=\raisebox{-2ex}{\shifttext{2ex}{$\lrcorner$}},
]{thmstyle}
\ifbook
    \declaretheorem[
        numbered=yes,
        style=thmstyle,
        within=chapter,
    ]{theorem}
\else
    \declaretheorem[
        numbered=yes,
        style=thmstyle,
    ]{theorem}
\fi
\declaretheoremstyle[
    headfont=\bfseries\itshape,
    bodyfont=\normalfont,
    notefont=,
    % headpunct={.\hfill\shifttext{2ex}{$\urcorner$}\break},
    headpunct={.\hfill\break},
    % headformat={\makebox[0pt][r]{\NAME\ \NUMBER}\NOTE},
    headformat={Thm\NOTE},
    qed=\raisebox{-2ex}{\shifttext{2ex}{$\lrcorner$}},
]{Thmstyle}
\declaretheorem[
    numbered=no,
    style=Thmstyle,
]{Theorem}

\declaretheoremstyle[
    headfont=\bfseries\itshape,
    bodyfont=\normalfont,
    notefont=,
    % headpunct={.\hfill\shifttext{2ex}{$\urcorner$}\break},
    headpunct={.\hfill\break},
    % headformat={\makebox[0pt][r]{\NAME\ \NUMBER}\NOTE},
    headformat={Cor \NUMBER \NOTE},
    qed=\raisebox{-2ex}{\shifttext{2ex}{$\lrcorner$}},
]{corstyle}
\ifbook
    \declaretheorem[
        numbered=yes,
        style=corstyle,
        within=chapter,
    ]{corollary}
\else
    \declaretheorem[
        numbered=yes,
        style=corstyle,
    ]{corollary}
\fi
\declaretheoremstyle[
    headfont=\bfseries\itshape,
    bodyfont=\normalfont,
    notefont=,
    % headpunct={.\hfill\shifttext{2ex}{$\urcorner$}\break},
    headpunct={.\hfill\break},
    % headformat={\makebox[0pt][r]{\NAME\ \NUMBER}\NOTE},
    headformat={Cor\NOTE},
    qed=\raisebox{-2ex}{\shifttext{2ex}{$\lrcorner$}},
]{Corstyle}
\declaretheorem[
    numbered=no,
    style=Corstyle,
]{Corollary}

\declaretheoremstyle[
    headfont=\bfseries\itshape,
    bodyfont=\normalfont,
    notefont=,
    % headpunct={.\hfill\shifttext{2ex}{$\urcorner$}\break},
    headpunct={.\hfill\break},
    % headformat={\makebox[0pt][r]{\NAME\ \NUMBER}\NOTE},
    headformat={\llap{\NAME} \NUMBER \NOTE},
    qed=\raisebox{-2ex}{\shifttext{2ex}{$\lrcorner$}},
]{theoremstyle}
\ifbook
    \declaretheorem[
        numbered=yes,
        style=theoremstyle,
        within=chapter,
    ]{proposition, lemma, claim}
\else
    \declaretheorem[
        numbered=yes,
        style=theoremstyle,
    ]{proposition, lemma, claim}
\fi
\declaretheorem[
    numbered=no,
    style=theoremstyle,
]{Proposition, Lemma, Claim}

% definitions
\declaretheoremstyle[
    headfont=\bfseries\itshape,
    bodyfont=\normalfont,
    notefont=,
    % headpunct={.\hfill\shifttext{2ex}{$\urcorner$}\break},
    headpunct={.\hfill\break},
    headformat={Def \NUMBER \NOTE},
    qed=\raisebox{-2ex}{\shifttext{2ex}{$\lrcorner$}},
]{defstyle}
\ifbook
    \declaretheorem[
        numbered=yes,
        style=defstyle,
        within=chapter,
    ]{definition}
\else
    \declaretheorem[
        numbered=yes,
        style=defstyle,
    ]{definition}
\fi
\declaretheoremstyle[
    headfont=\bfseries\itshape,
    bodyfont=\normalfont,
    notefont=,
    % headpunct={.\hfill\shifttext{2ex}{$\urcorner$}\break},
    headpunct={.\hfill\break},
    headformat={Def\NOTE},
    qed=\raisebox{-2ex}{\shifttext{2ex}{$\lrcorner$}},
]{Defstyle}
\declaretheorem[
    numbered=no,
    style=Defstyle,
]{Definition}

\declaretheoremstyle[
    headfont=\bfseries\itshape,
    bodyfont=\normalfont,
    notefont=,
    % headpunct={.\hfill\shifttext{2ex}{$\urcorner$}\break},
    headpunct={.\hfill\break},
    headformat={Axiom \NUMBER \NOTE},
    qed=\raisebox{-2ex}{\shifttext{2ex}{$\lrcorner$}},
]{axmstyle}
\ifbook
    \declaretheorem[
        numbered=yes,
        style=axmstyle,
        % within=chapter,
    ]{axiom}
\else
    \declaretheorem[
        numbered=yes,
        style=axmstyle,
    ]{axiom}
\fi
\declaretheoremstyle[
    headfont=\bfseries\itshape,
    bodyfont=\normalfont,
    notefont=,
    % headpunct={.\hfill\shifttext{2ex}{$\urcorner$}\break},
    headpunct={.\hfill\break},
    headformat={Axiom\NOTE},
    qed=\raisebox{-2ex}{\shifttext{2ex}{$\lrcorner$}},
]{Axmstyle}
\declaretheorem[
    numbered=no,
    style=Axmstyle,
]{Axiom}

\declaretheoremstyle[
    headfont=\bfseries\itshape,
    bodyfont=\normalfont,
    notefont=,
    % headpunct={.\hfill\shifttext{2ex}{$\urcorner$}\break},
    headpunct={.\hfill\break},
    headformat={\llap{\NAME} \NUMBER \NOTE},
    qed=\raisebox{-2ex}{\shifttext{2ex}{$\lrcorner$}},
]{definitionstyle}
\ifbook
    \declaretheorem[
        numbered=yes,
        style=definitionstyle,
        within=chapter,
    ]{algorithm, notation}
\else
    \declaretheorem[
        numbered=yes,
        style=definitionstyle,
    ]{algorithm, notation}
\fi
\declaretheorem[
    numbered=no,
    style=definitionstyle,
]{Algorithm, Notation}

% examples
\declaretheoremstyle[
    headfont=\bfseries\itshape,
    bodyfont=\normalfont,
    notefont=\bfseries\itshape,
    headpunct={.\hfill\break},
]{examplestyle}
\ifbook
    \declaretheorem[
        numbered=yes,
        style=examplestyle,
        within=chapter,
    ]{exercise, example, ex, counterexample}
\else
    \declaretheorem[
        numbered=yes,
        style=examplestyle,
    ]{exercise, example, ex, counterexample}
\fi
\declaretheorem[
    numbered=no,
    style=examplestyle,
]{Exercise, Example, Ex, Counterexample}

% notes
\declaretheoremstyle[
    headfont=\bfseries\itshape,
    bodyfont=\normalfont,
    notefont=\bfseries\itshape,
    headformat={\llap{\NAME} \NUMBER \NOTE},
]{notestyle}
\ifbook
    \declaretheorem[
        numbered=yes,
        style=notestyle,
        within=chapter,
    ]{note, remark, idea, intuition}
\else
    \declaretheorem[
        numbered=yes,
        style=notestyle,
    ]{note, remark, idea, intuition}
\fi
\declaretheorem[
    numbered=no,
    style=notestyle,
]{Note, Remark, Idea, Intuition}

% \newenvironment{case}[1][Case]
    % {\textit{#1:}\quote}
    % {\endquote}
\newenvironment{case}[1][Case]
    {\par\textit{#1:}\hfill\break}
    {}

\newenvironment{code}[1][Code]
    {\textbf{#1:}\quote}
    {\endquote}

% \newcommand{\concept}{\textbf{\textit{Intuition.}}\ }

% \def\lstlistingautorefname{Algorithm}
% \def\itemautorefname{Section}
% \renewcommand{\chapterautorefname}{Chapter}
% \renewcommand{\sectionautorefname}{Section}
% \renewcommand{\theoremautorefname}{Theorem}
% \newcommand{\axiomautorefname}{Axiom}
% \newcommand{\lemmaautorefname}{Lemma}
% \newcommand{\propositionautorefname}{Proposition}
% \newcommand{\corollaryautorefname}{Corollary}
% \newcommand{\claimautorefname}{Claim}
% \newcommand{\conjectureautorefname}{Conjecture}
% \newcommand{\justificationautorefname}{Justification}
% \newcommand{\definitionautorefname}{Definition}
% \newcommand{\notationautorefname}{Notation}
% \newcommand{\exampleautorefname}{Example}
% \newcommand{\counterexampleautorefname}{Counterexample}
% \newcommand{\noteautorefname}{Note}
% \newcommand{\remarkautorefname}{Remark}
% \newcommand{\ideaautorefname}{Idea}
% \newcommand{\intuitionautorefname}{Intuition}
