% new proof environment
\expandafter\let\expandafter\oldproof\csname\string\proof\endcsname
\let\oldendproof\endproof
\renewenvironment{proof}[1][\bfseries\proofname]{%
  \vspace{\ProofSpacing}%
  \oldproof[\bfseries #1]
}{\qed}

% style for theorems
\declaretheoremstyle[headfont=\bfseries\itshape,
                     bodyfont=\normalfont,
                     notefont=,
                     headpunct={.\hfill\break},
                     qed=\kanji{定理},  % 定理結束
                     % prefoothook={\hfill\break}
                     ]{theoremstyle}
\declaretheoremstyle[headfont=\bfseries\itshape,
                     bodyfont=\normalfont,
                     notefont=,
                     headpunct={.\hfill\break},
                     qed=\kanji{推論},  % 推論結束
                     % prefoothook={\hfill\break}
                     ]{corollarystyle}
\declaretheoremstyle[headfont=\bfseries\itshape,
                     bodyfont=\normalfont,
                     notefont=,
                     headpunct={.\hfill\break},
                     qed=\kanji{引理},  % 引理結束
                     % prefoothook={\hfill\break}
                     ]{lemmastyle}
\declaretheoremstyle[headfont=\bfseries\itshape,
                     bodyfont=\normalfont,
                     notefont=,
                     headpunct={.\hfill\break},
                     qed=\kanji{定義},  % 定義結束
                     % prefoothook={\hfill\break}
                     ]{definitionstyle}
\declaretheoremstyle[headfont=\bfseries\itshape,
                     bodyfont=\normalfont,
                     notefont=,
                     headpunct={.\hfill\break},
                     qed=\kanji{公理},  % 公理結束
                     % prefoothook={\hfill\break}
                     ]{axiomstyle}
\declaretheoremstyle[headfont=\bfseries\itshape,
                     bodyfont=\normalfont,
                     notefont=\bfseries\itshape,
                     headpunct={.\hfill\break},
                     qed=\kanji{範例},  % 範例結束
                     % prefoothook={\hfill\break}
                     ]{examplestyle}
\declaretheoremstyle[headfont=\bfseries\itshape,
                     bodyfont=\normalfont,
                     notefont=\bfseries\itshape,
                     headpunct={.\hfill\break},
                     qed=\kanji{演算法},  % 演算法結束
                     % prefoothook={\hfill\break}
                     ]{algorithmstyle}
\declaretheoremstyle[headfont=\bfseries\itshape,
                     bodyfont=\normalfont,
                     notefont=\bfseries\itshape,
                     qed=\kanji{直覺},  % 直覺結束
                     % prefoothook={\hfill\break}
                     ]{ideastyle}

\ifbeamer
\else
  \declaretheorem[numbered=yes,style=theoremstyle,within=\BookChapterOption]{theorem}
  \declaretheorem[numbered=yes,style=corollarystyle,within=\BookChapterOption]{corollary}
  \declaretheorem[numbered=yes,style=lemmastyle,within=\BookChapterOption]{lemma}
  \declaretheorem[numbered=yes,style=definitionstyle,within=\BookChapterOption]{definition}
  \declaretheorem[numbered=yes,style=axiomstyle]{axiom}
  \declaretheorem[numbered=yes,style=examplestyle,within=\BookChapterOption]{exercise, example, counterexample}
  \declaretheorem[numbered=yes,style=algorithmstyle,within=\BookChapterOption]{algorithm}
  \declaretheorem[numbered=yes,style=ideastyle,within=\BookChapterOption]{idea}

  \declaretheorem[numbered=no,style=theoremstyle]{Theorem}
  \declaretheorem[numbered=no,style=corollarystyle]{Corollary}
  \declaretheorem[numbered=no,style=lemmastyle]{Lemma}
  \declaretheorem[numbered=no,style=definitionstyle]{Definition}
  \declaretheorem[numbered=no,style=axiomstyle]{Axiom}
  \declaretheorem[numbered=no,style=examplestyle]{Exercise, Example, Counterexample}
  \declaretheorem[numbered=no,style=algorithmstyle]{Algorithm}
  \declaretheorem[numbered=no,style=ideastyle]{Idea}
\fi

\crefname{theorem}{\emph{theorem}}{\emph{Theorem}}
\crefname{corollary}{\emph{corollary}}{\emph{Corollary}}
\crefname{lemma}{\emph{lemma}}{\emph{Lemma}}
\crefname{definition}{\emph{definition}}{\emph{Definition}}
\crefname{axiom}{\emph{axiom}}{\emph{Axiom}}
\crefname{exercise}{exercise}{Exercise}
\crefname{example}{example}{Example}

\crefname{figure}{\emph{figure}}{\emph{Figure}}
\crefname{table}{\emph{table}}{\emph{Table}}

\newenvironment{code}[1][Code]
  {\textbf{#1:}\quote}
  {\endquote}

\ifminted
  \renewcommand{\theFancyVerbLine}{\normalfont\textcolor{foreground!50!background}{\tiny\arabic{FancyVerbLine}}}

  % \newtcblisting{code}[1][]{
    % verbatim ignore indention at end=true,
    % left=3pt, bottom=0pt, top=0pt, right=3pt,
    % colframe=foreground, colback=foreground,
    % listing only,
    % on line,
    % #1,
  % }

  \newtcblisting{codedark}[2][]{
    listing engine=minted,
    minted options app={escapeinside=??,mathescape,linenos,fontsize=\footnotesize},  
    % minted options app={escapeinside=??,mathescape,linenos,fontsize=\normalsize},
    minted style=coffee,
    minted language=#2,
    verbatim ignore indention at end=true,
    left=3pt, bottom=0pt, top=0pt, right=3pt,
    colframe=foreground, colback=foreground, colupper=butter,
    listing only,
    on line,
    #1,
  }

  \newtcblisting{codelight}[2][]{
    listing engine=minted,
    minted options app={escapeinside=??,mathescape,linenos,fontsize=\footnotesize},  
    % minted options app={escapeinside=??,mathescape,linenos,fontsize=\normalsize},
    minted style=gruvbox-light,
    minted language=#2,
    verbatim ignore indention at end=true,
    left=3pt, bottom=0pt, top=0pt, right=3pt,
    colframe=butter, colback=butter,
    listing only,
    on line,
    #1,
  }

  \NewTotalTCBox{\codeindark}{O{text} v}{
    before upper=\strut,
    boxsep=0pt, left=3pt, bottom=0pt, top=0pt, right=3pt,
    colframe=foreground, colback=foreground,
    listing only,
    on line,
  }{\mintinline[escapeinside=??,mathescape,style=coffee,fontsize=\footnotesize]{#1}{#2}}
  % }{\mintinline[escapeinside=??,mathescape,style=coffee,fontsize=\normalsize]{#1}{#2}}

  \NewTotalTCBox{\codeinlight}{O{text} v}{
    before upper=\strut,
    boxsep=0pt, left=3pt, bottom=0pt, top=0pt, right=3pt,
    colframe=butter, colback=butter,
    listing only,
    on line,
  }{\mintinline[escapeinside=??,mathescape,style=gruvbox-light,fontsize=\footnotesize]{#1}{#2}}
  % }{\mintinline[escapeinside=??,mathescape,style=gruvbox-light,fontsize=\normalsize]{#1}{#2}}
\else
\fi

% \ifblank{#1}{\:\cdot\:}{#1}

\newenvironment{case}[1][]
  % {\par\textit{#1:}\hfill\break}
  {\quote\textit{\ifblank{#1}{}{#1:\hfill\break}}}
  % {\par\begin{mdframed}[backgroundcolor=background,
      % linecolor=background,
      % % hidealllines,
      % fontcolor=foreground,
      % roundcorner=5pt]%
  % \begin{mdframed}[backgroundcolor=foreground, fontcolor=background, roundcorner=5pt]
    % \textit{#1:}}
  % {\end{mdframed}
  {\endquote}

% align-optional cases environment
\makeatletter
  \renewenvironment{cases}[1][l]{\matrix@check\cases\env@cases{#1}}{\endarray\right.}
  \def\env@cases#1{%
    \let\@ifnextchar\new@ifnextchar
    \left\lbrace\def\arraystretch{1.2}%
    \array{@{}#1@{~}l@{}}}
\makeatother

% TODO: axe this
% \ifalgorithms
  % \newcounter{nalg}[chapter]
  % \renewcommand{\thenalg}{\thechapter.\arabic{nalg}}
  % \DeclareCaptionLabelFormat{algocaption}{\it Algorithm \thenalg}

  % \lstnewenvironment{algorithm}[1][]
  % {
    % \refstepcounter{nalg}
    % \captionsetup{labelformat=algocaption,labelsep=colon}
    % \lstset{mathescape=true,
            % frame=tB,
            % numbers=left,
            % numberstyle=\tiny,
            % basicstyle=\scriptsize,
            % keywordstyle=\color{black}\bfseries\em,
            % keywords={,input, output, return, datatype, function, in, if, else, elif, for, foreach, while, not, begin, end, true, false, null, break, continue, let, and, or, }
            % numbers=left,
            % xleftmargin=.04\textwidth,
            % #1}
  % }
  % {}
% \else
% \fi
